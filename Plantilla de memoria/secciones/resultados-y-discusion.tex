% RESULTADOS Y DISCUSION 

\cleardoublepage

\chapter{Resultados y Discusión}
\label{resultados-y-discusion}

\section{Resumen de resultados}
En esta secci\'on se resumen los resultados obtenidos en los experimentos y se discuten los patrones observados. Para facilitar la comparaci\'on entre modelos, se reportan m\'etricas globales (agregadas) y m\'etricas por tienda. La discusi\'on se centra en dos ejes: (i) el impacto de incorporar variables ex\'ogenas y (ii) la capacidad de generalizaci\'on de modelos globales frente a enfoques por serie.

\section{Comparación global entre familias (configuración base)}
Como punto de partida, la Tabla~\ref{tab:comparacion-global} compara familias de modelos con una configuración base.

% TABLA: COMPARACIÓN GLOBAL ENTRE FAMILIAS (CONFIGURACIÓN BASE)

\begin{table}[ht!]
\centering
\caption[Comparación global (MAE/RMSE/sMAPE)]{\textbf{Comparación global entre modelos (configuración base).} Métricas agregadas sobre todas las tiendas y semanas del horizonte de test. Menor es mejor.}
\label{tab:comparacion-global}
\resizebox{\textwidth}{!}{%
\begin{tabular}{lrrr}
\toprule
Modelo & MAE & RMSE & sMAPE \\
\midrule
DeepAR (exógenas) & 59\,573.68 & 86\,287.62 & 6.041 \\
LSTM global (exógenas) & 75\,827.53 & 98\,575.42 & 8.016 \\
Prophet (regresores) & 142\,509.72 & 186\,757.80 & 12.139 \\
SARIMAX (exógenas) & 124\,157.68 & 169\,022.47 & 10.889 \\
Transformer global (exógenas) & 113\,420.92 & 147\,263.71 & 9.863 \\
\bottomrule
\end{tabular}%
}
\end{table}


\noindent La Figura~\ref{fig:metrics-comparison-global} complementa la tabla, facilitando una lectura visual conjunta.

\begin{figure}[ht!]
	\centering
	\includegraphics[width=0.95\textwidth]{figuras/metrics_comparison_global.png}
	\caption[Comparación global de métricas]{\textbf{Comparación global de métricas (MAE/RMSE/sMAPE).} Menor es mejor.}
	\label{fig:metrics-comparison-global}
\end{figure}

\section{Ranking global por MAE}
La Figura~\ref{fig:top10-mae} muestra el ranking de los mejores modelos seg\'un \acrshort{mae}. Esta visualizaci\'on es \''util para identificar configuraciones competitivas y comparar la estabilidad de rendimiento.

\begin{figure}[ht!]
		\centering
		\includegraphics[width=0.95\textwidth]{figuras/top10_mae.png}
		\caption[Top 10 modelos por MAE]{\textbf{Top 10 modelos por MAE.} Menor es mejor.}
		\label{fig:top10-mae}
\end{figure}

\section{Tabla comparativa de m\'etricas}
La Tabla~\ref{tab:top10} recoge \acrshort{mae}, \acrshort{rmse} y \acrshort{smape} para los modelos mejor posicionados seg\'un el resumen experimental.

\begin{table}[ht!]
\centering
\small
\resizebox{\textwidth}{!}{%
\begin{tabular}{p{9.5cm}rrr}
\toprule
Modelo & MAE & RMSE & sMAPE \\
\midrule
transformer\_exog\_\_dm128\_\_nh8\_\_nl4\_\_do0.1\_\_lr0.0003 & 50,989.60 & 68,451.73 & 5.53 \\
deepar\_exog\_\_hs80\_\_nl2\_\_do0.1\_\_lr0.0005\_\_bs32 & 58,954.49 & 82,957.67 & 6.16 \\
deepar\_exog & 59,573.68 & 86,287.62 & 6.04 \\
deepar\_exog\_\_hs40\_\_nl2\_\_do0.1\_\_lr0.001\_\_bs32 & 61,163.80 & 89,421.24 & 6.13 \\
deepar\_exog\_\_hs80\_\_nl3\_\_do0.2\_\_lr0.001\_\_bs64 & 62,260.56 & 89,886.14 & 6.19 \\
deepar\_exog\_\_hs40\_\_nl3\_\_do0.2\_\_lr0.0005\_\_bs64 & 62,739.17 & 89,809.35 & 6.42 \\
transformer\_exog\_\_dm64\_\_nh4\_\_nl2\_\_do0.2\_\_lr0.0003 & 64,173.73 & 85,342.68 & 6.69 \\
transformer\_exog\_\_dm128\_\_nh8\_\_nl2\_\_do0.2\_\_lr0.001 & 82,876.35 & 104,778.98 & 11.05 \\
transformer\_exog\_\_dm64\_\_nh4\_\_nl2\_\_do0.1\_\_lr0.001 & 87,233.72 & 105,921.99 & 10.93 \\
sarimax\_exog & 124,157.68 & 169,022.47 & 10.89 \\
\bottomrule
\end{tabular}
}
\caption{\textbf{Top-10 modelos globales por MAE (E2).}}
\label{tab:top10}
\end{table}


\section{Comparaci\'on normalizada de las top-5 configuraciones}
La Figura~\ref{fig:top5-normalized} compara las top-5 configuraciones tras normalizar cada m\'etrica para facilitar una lectura conjunta. En general, un modelo puede dominar en \acrshort{mae} pero no necesariamente en \acrshort{rmse} (m\'as sensible a errores grandes) o en \acrshort{smape} (m\'etrica relativa).

\begin{figure}[ht!]
		\centering
		\includegraphics[width=0.95\textwidth]{figuras/top5_metrics_normalized.png}
		\caption[Top 5 comparaci\'on normalizada]{\textbf{Top 5: comparaci\'on normalizada de MAE, RMSE y sMAPE.} Menor es mejor.}
		\label{fig:top5-normalized}
\end{figure}

\section{Discusi\'on}
De forma cualitativa, se observan los siguientes patrones:
\begin{itemize}
	\item \textbf{Beneficio de covariables}: las variables de calendario tienden a aportar se\~nal consistente; su impacto es mayor en tiendas con estacionalidad marcada.
	\item \textbf{Modelos globales}: en presencia de m\'ultiples tiendas, los enfoques globales (DeepAR, \acrshort{lstm} y Transformer) pueden beneficiarse de patrones compartidos, en especial cuando algunas series son cortas.
	\item \textbf{Interpretabilidad}: \acrshort{sarimax} y Prophet aportan interpretabilidad y diagn\'ostico de componentes. En cambio, los modelos neuronales capturan no linealidades y efectos combinados, a costa de mayor complejidad.
	\item \textbf{Sensibilidad a cambios de r\'egimen}: los errores tienden a aumentar ante cambios estructurales; este aspecto se analiza en experimentos de robustez y se retoma en limitaciones.
\end{itemize}

En conjunto, los resultados respaldan el uso de variables ex\'ogenas cuando su disponibilidad a futuro est\'a garantizada o es estimable, y sugieren que el aprendizaje global es una estrategia adecuada en escenarios multiserie.

\section{Ablaciones: valor de las exógenas y del calendario}
Una cuestión clave es separar el aporte de:
\begin{itemize}
	\item \textbf{Calendario} (conocido a futuro), y
	\item \textbf{Exógenas observadas} (no garantizadas a futuro sin hipótesis adicional).
\end{itemize}
\noindent [COMPLETAR: tabla/figura de ablación con 3 condiciones: (i) solo lags, (ii) lags+calendario, (iii) lags+calendario+exógenas; reportar MAE/RMSE/sMAPE global y por tienda.]

\section{Escenario realista vs. escenario \emph{oracle}}
La hipótesis \emph{oracle} (exógenas conocidas en el horizonte) tiende a favorecer modelos con regresores ricos.
En un despliegue realista, la comparación debe considerar o bien exógenas realmente conocidas (calendario, festivos), o bien el error compuesto de un pipeline que primero predice exógenas y después predice ventas.
\noindent [COMPLETAR: experimento adicional con predicción de exógenas o con sustitución por "last observation carried forward" para exógenas no conocidas.]

\section{Heterogeneidad por tienda}
Además de la métrica global, el análisis por tienda permite identificar:
\begin{itemize}
	\item tiendas donde los modelos globales transfieren bien patrones,
	\item tiendas con alta volatilidad o cambios de régimen,
	\item casos donde un modelo simple puede ser competitivo.
\end{itemize}
\noindent [COMPLETAR: incluir percentiles (p50/p75/p90) de MAE por tienda y ejemplos de 2--3 tiendas representativas.]
