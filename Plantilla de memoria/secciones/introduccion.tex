% INTRODUCCIÓN

\cleardoublepage

\chapter{Introducción}
\label{introduccion}

La predicci\'on de demanda es una pieza central en la gesti\'on de cadenas de suministro y operaciones minoristas: afecta a decisiones de inventario, planificaci\'on de personal, reposici\'on y dise\~no de promociones. En series temporales reales, las ventas presentan estacionalidad (anual y semanal), efectos calendario (festivos y eventos), tendencias, as\'i como perturbaciones externas asociadas a cambios macroecon\'omicos o a choques de oferta y demanda. Este Trabajo Fin de M\'aster se centra en el pron\'ostico de ventas semanales por tienda, utilizando informaci\'on hist\'orica y un conjunto de variables ex\'ogenas.

\section{Motivaci\'on}
En entornos multi-tienda es habitual disponer de decenas o centenares de series con comportamientos heterog\'eneos. Modelar cada tienda de forma independiente puede ser sub\'optimo cuando hay series cortas o ruidosas; en cambio, los \emph{modelos globales} explotan patrones compartidos y pueden mejorar la generalizaci\'on. Al mismo tiempo, existe una necesidad pr\'actica de incorporar covariables externas (por ejemplo, festivos o indicadores econ\'omicos), lo que favorece enfoques que acepten regresores.

\section{Planteamiento del problema}
Se aborda un problema de predicci\'on multiserie: para cada tienda, se desea predecir las ventas en un horizonte fijo (semanas futuras) a partir del historial de ventas y covariables. El proyecto prioriza una comparaci\'on \emph{homog\'enea} de modelos: se seleccionan m\'etodos capaces de manejar variables ex\'ogenas y se define un protocolo de validaci\'on temporal com\'un para todos.

\section{Aportaciones}
Las principales contribuciones del trabajo son:
\begin{itemize}
  \item Un protocolo experimental reproducible para comparar modelos con regresores ex\'ogenos en ventas semanales.
  \item Una implementaci\'on unificada de experimentos y m\'etricas (\acrshort{mae}, \acrshort{rmse}, \acrshort{smape}) con resultados globales y por tienda.
  \item Un an\'alisis de robustez ante series cortas y ante cambios estructurales asociados a un choque econ\'omico.
  \item Integraci\'on de modelos representativos: \acrshort{sarimax}, Prophet \citep{taylor2018forecasting}, DeepAR \citep{flunkert2017deepar} y arquitecturas neuronales globales (\acrshort{lstm} \citep{hochreiter1997long} y Transformer \citep{vaswani2017attention}).
\end{itemize}

\section{Estructura de la memoria}
El documento se organiza de la siguiente forma. El cap\'itulo de \textbf{Estado del arte} revisa los enfoques relevantes para series temporales con covariables. El \textbf{Marco te\'orico} introduce los conceptos necesarios (validaci\'on temporal, fuga de informaci\'on, m\'etricas y regularizaci\'on). A continuaci\'on, se definen los \textbf{Objetivos}, la \textbf{Metodolog\'ia} y el dise\~no experimental. Finalmente, se presentan \textbf{Resultados y Discusi\'on}, \textbf{Conclusiones} y \textbf{Limitaciones y l\'ineas futuras}.
