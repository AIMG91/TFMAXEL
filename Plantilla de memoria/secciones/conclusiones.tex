% CONCLUSIONES

\chapter{Conclusiones}
\label{conclusiones}

\begin{enumerate}[label=\destacado{\arabic*.}]
  \setlength\itemsep{1em}
  \item La incorporaci\'on de variables ex\'ogenas mejora el pron\'ostico cuando dichas covariables aportan se\~nal predictiva y son coherentes con el horizonte de predicci\'on.

  \item La comparaci\'on bajo un protocolo temporal \emph{com\'un} (particiones consistentes y caracter\'isticas causales) es esencial para evitar conclusiones sesgadas por fuga de informaci\'on.

  \item Los modelos globales (DeepAR y redes neuronales entrenadas con m\'ultiples tiendas) tienden a beneficiarse del aprendizaje compartido, especialmente en tiendas con menos historial.

  \item Modelos estad\'isticos como \acrshort{sarimax} y enfoques aditivos como Prophet ofrecen una relaci\'on favorable entre interpretabilidad y rendimiento, resultando opciones robustas para entornos con restricciones operativas.

  \item Los experimentos de robustez sugieren que los cambios estructurales pueden degradar significativamente el rendimiento, lo que motiva estrategias de monitorizaci\'on y reentrenamiento.
\end{enumerate}
